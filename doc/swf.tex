\documentclass[fleqn]{article}
\author{MODFLOW Development Team}

\usepackage{amsmath}
\usepackage{xcolor}
\usepackage{graphicx}
\graphicspath{{./figures/}}


\begin{document}

\title{Surface Water Flow in MODFLOW 6}
\maketitle

\tableofcontents

\section{Introduction}
The Surface Water Flow (SWF) Model simulates one-dimensional channel flow and two-dimensional overland flow.  The SWF Model is a new model type in the MODFLOW 6 framework.  Like other models in the MODFLOW 6 framework, the SWF Model consists of required and optional packages.

Shallow water equations
Vertically integrated surface water flow

\section{Formulation and Solution of the Control-Volume Finite-Difference Equation}

The surface water flow equation in MODFLOW 6 is discretized using a control-volume finite-difference (CVFD) method. This section describes the mathematical equations discretized in the SWF model, discretization options, and the general forms of the finite-difference equations used to simulate channel and overland surface water flow.

\subsection{Mathematical Model}
Following~\cite{panday2004} and~\cite{hughes2012documentation} the partial differential equation for the the diffusive wave approximation to the full St. Venant equations can be written for both one-dimensional channel flow and two-dimensional overland flow.  The following equation is for one-dimensional channel flow,

\begin{equation}
  \frac{\partial}{\partial \ell}
  \left (
  \frac{A R^{\frac{2}{3}}}{n \left | \frac{\partial h}{\partial \ell} \right |^{\frac{1}{2}} } \frac{\partial h}{\partial \ell}
  \right )
  + Q_s'
  =   
  \frac{\partial A}{\partial t},
\label{eqn:onedpd}
\end{equation}

\noindent where $\ell$ is the length ($L$) along the channel, $A$ is the cross-sectional area  ($L^2$) for flow, $R$ is the hydraulic radius ($L$), $n$ is the Manning's roughness coefficient ($T/L^{\frac{1}{3}}$), $h$ is the water surface elevation ($L$), $Q_s'$ is a source term representing the volumetric flow rate of a source per length of channel ($L^2/T$), and $t$ is time ($T$). 

For two-dimensional overland flow, the partial differential equation can be written as

\begin{equation}
  \frac{\partial}{\partial x}
  \left (
  \frac{d^{\frac{5}{3}}}{n \left | \frac{\partial h}{\partial s} \right |^{\frac{1}{2}} } \frac{\partial h}{\partial x}
  \right )
  + \frac{\partial}{\partial y}
  \left (
  \frac{d^{\frac{5}{3}}}{n \left | \frac{\partial h}{\partial s} \right |^{\frac{1}{2}} } \frac{\partial h}{\partial y}
  \right )
  + q_s
  = \frac{\partial h}{\partial t},
  \label{eqn:twodpd}
\end{equation}

\noindent where $x$ and $y$ are the two-dimensional coordinate directions, $d$ is the water depth ($L$), $s$ is the coordinate direction corresponding to the maximum water surface slope ($L$), and $q_s$ is a source term, in dimensions of $L/T$, representing the volumetric rate of source water per unit area.

Due to the dimensionality difference between one-dimensional channel flow and two-dimensional overland flow, the dimensions of the terms in equations~\ref{eqn:onedpd} and~\ref{eqn:twodpd} are different.  Equation~\ref{eqn:onedpd} has dimensions of $L^2/T$ whereas equation~\ref{eqn:twodpd} has dimensions of $L/T$.

\subsection{Control-Volume Finite-Difference Method}
Equations~\ref{eqn:onedpd} and~\ref{eqn:twodpd}, together with initial conditions and any relevant boundary conditions, represents mathematically the surface water flow balance at any point in the model domain. In certain simple cases, equations~\ref{eqn:onedpd} and~\ref{eqn:twodpd} can be solved analytically to obtain a mathematical expression for water surface elevation and flow throughout a model domain. The balance of surface water flow is formulated for each model cell, taking into account the flows of water to and from neighboring cells, as well as changes in storage and any external sources and sinks.  Taken together, the surface water flow balance equations for all the cells form a system of linear equations that is solved iteratively using a linear matrix solver. Details of the CVFD implementation in the SWF Model are described below.

\subsection{Characteristics of a Model Cell}
An SWF Model can represent either one-dimensional channel flow or two-dimensional overland flow.  The characterics of a model cell are different depending on whether the model represents channel flow or overland flow.

For one-dimensional channels, surface water flows through a network of connected model cells.  Model cells are one dimensional in plan view but can have different cross-sectional geometries.  The default cross-sectional geometry corresponds to a hydraulically wide rectangular channel in which there is no resistance to flow along the vertical sides.  If the default channel configuration does not represent field conditions, then the Cross Section (CXS) Package can be activated and used to provide more detailed cross-section information.

Two-dimensional overland flow is modeled on a structured or unstructured grid.  For each cell, the SWF Model calculates the water surface elevation for the cell and for flow between each model cell and its neighbors.  Resistance to flow occurs through a Manning's roughness coefficient.

\subsection{Connections Between Cells}
A connection between two cells indicates that the cells are in hydraulic communication where surface water can flow between the cells under the influence of a hydraulic gradient.  Cell connections are calculated by the SWF Model using information about the channel network or the overland flow grid.  For channel (DISV1D) and two dimensional vertex (DISV2D) grids, cell connectivity is determined by the presence of shared vertices.  For a structured two-dimensional grid (DIS2D), each cell is hydraulically connected to the four surrounding cells.  

For accurate solutions with the CVFD method, there are geometric requirements regarding cell connections.  First, a line drawn between the centers of any two connected cells should intersect the shared face at a right angle. Second, the intersection point should coincide with an appropriate mean position on the shared face \cite{narasimhan1976integrated}.  The smaller the deviation from these CVFD connection requirements, the smaller the loss in accuracy in the surface water flow solution. 

\subsection{Model Grids}
The SWF Model supports two different types of discretization approaches: a one-dimensional channel network and a two-dimensional overland flow grid.  The two-dimensional overland flow grid may use either a structured grid consisting of rows and columns or an unstructured grid specified using vertices and cells that are defined using those vertices.  A SWF Model must use either the one-dimensional channel network or the the two-dimensional overland flow discretization approaches.  

\subsubsection{One-Dimensional Channel Flow}
A one-dimensional channel network is represented using the Discretization by Vertices in One Dimension (DISV1D) Package.  An example of a channel network is shown in figure~\ref{fig:channel_network}.  The channel network is discretized into reaches.  The SWF Model solves for the water level in reach by balancing the flows between reaches with storage changes and sources.  Stage calculation points within a reach are typically located halfway between the reach starting and ending vertices; however, the user may adjust the stage calculation point in either direction (toward or away from the first reach vertex), which can be advantageous in some situations.  A reach is connected to another reach when they share a vertex.  Reaches do not need to be defined in any particular order.  If a reach terminates at a vertex and that vertex is not shared by another reach, then by default, there is no outflow from the ending vertex location.  If the user intends for outflow to occur, then a stress package must be activated and assigned to that reach.

\begin{figure}
  \centering
  \includegraphics[scale=0.9]{figures/channel_network.pdf}
  \caption[Schematic showing a channel network.]{Schematic showing a channel network.  The channel network is discretized by the user into reaches.  Stage is calculated within the reach at stage calculation points.  Stage calculations are typically located halfway between the reach start and reach end vertices, however, the user can adjust the fractional position of the stage calculation point along the reach.}
  \label{fig:channel_network}
\end{figure}
  
\subsubsection{Two-Dimensional Overland Flow}
Two-dimensional overland flow can be represented using either the regular structured grid discretization (DIS2D) Package or the unstructured discretization by vertices (DISV2D) Package.  These two discretization types are patterned after their Groundwater Flow (GWF) Model counterparts, DIS and DISV, respectively.  The difference between the SWF overland flow discretization types and the GWF Model discretization types is that the SWF versions are two dimensional and do not have a top defined or multiple layers.

For DIS2D grids, stage calculation points are located in the center of the model cell.  For DISV2D grids, stage calculation points are assigned by the user; they should generally be located in the center of the cell, but there is flexibility to adjust them spatially to better represent precise boundary locations.  By default, the model domain boundary represents no-flow conditions.  Outflow conditions must be activated by assigning a stress package to the desired locations.

\subsection{Internal Flows Between Cells}
Flows between cells are called ``internal flows.'' Internal flows between cells are calculated using Manning's equation.  The following is a general expression for Manning's equation:

\begin{equation}
  Q = \frac{1}{n} A R^{\frac{2}{3}} \sqrt{S_f},
\end{equation}

\noindent where $R$ is the hydraulic radius ($L$) equal to the flow area divided by the wetted perimeter, and $S_f$ is the water surface slope (dimensionless). 

For the numerical implementation in MODFLOW we express Manning's equation as a product of a conductance term and the head difference between reach $n$ and $m$:

\begin{equation}
  Q_{nm} = \overline{C}_{nm} \left ( h_m - h_n \right ).
\end{equation}

\noindent The conductance-based flow expression allows individual cell conductances to be calculated for each of the two connected reaches and then averaged together to give a representative conductance for the connection.  The representative conductance between reach $n$ and reach $m$ ($\overline{C}_{nm}$) is calculated by averaging a conductance for reach $n$ with a conductance for reach $m$.  The conductance for reach $n$, denoted by $C_{mn}$ ($L^2/T$), is calculated over the distance from the stage calculation point of $n$ to its shared face with cell $m$.  Likewise, the conductance for reach $m$, over the distance between the stage calculation point for $m$ and its shared face with cell $n$ is denoted by $C_{mn}$ and also has dimensions of $L^2/T$.

The average conductance between reach $n$ and reach $m$ is calculated using the harmonic mean of $C_{nm}$ and $C_{mn}$ to give

\begin{equation}
  \overline{C}_{nm} = \frac{C_{nm}  C_{mn}}{C_{nm} + C_{mn}}.
\end{equation}

\noindent The harmonic mean preferentially weights the average toward the lower of the two values and is based on piecewise constant reach properties that may change abruptly at the face between different reaches.

The conductance for reach $n$ in the $m$ direction ($C_{nm}$) can be expressed as

\begin{equation}
  C_{mn} = 
  \frac{
  A_n 
  R_{n}^{\frac{2}{3}}
  }
  {n_n
  L_{nm}
  \sqrt{| \gamma_n |}
  },
\label{eqn:cn}
\end{equation}

\noindent where $A_n$ is the cross-sectional flow area ($L^2$) for reach $n$ calculated as a function of the water depth, $R_n$ is the hydraulic radius ($L$) for reach $n$, which is the cross-sectional flow area divided by the wetted perimeter, calculated as a function of the water depth, $n_n$ is the Manning's roughness coefficient ($T/L^{\frac{1}{3}}$) for reach $n$,  $L_{nm}$ is the distance ($L$) from the stage calculation point in cell $n$ to the shared edge with reach $m$, and $\gamma_n$ is the hydraulic gradient ($L^0$) for reach $n$.

Several of the terms in equation \ref{eqn:cn} are combined into a single term for reach $n$ called channel conveyance ($\kappa_n$), which is defined as

\begin{equation}
  \kappa_n = \frac{A_n R_n^{\frac{2}{3}}}{n_n}.
\label{eqn:conveyance}
\end{equation}

\noindent Channel conveyance for reach $n$ is calculated as a function of an upstream-weighted water depth.  Thus, channel conveyance for reach $n$ can be written as

\begin{equation}
  \kappa_n = \frac{A_n (d_u) R_n (d_u) ^{\frac{2}{3}}}{n_n},
\label{eqn:conveyancedu}
\end{equation}

\noindent where the $A_n$ and $R_n$ terms include $(d_u)$ to indicate that they are calculated as a function of $d_u$.  $d_u$ is assigned to the depth of the reach (either from cell $n$ or cell $m$) with the higher stage:

\[
d_u = 
\begin{cases}
  d_n & \text{if $h_n>h_m$} \\
  d_m & \text{otherwise}
\end{cases}.
\]

With this definition for channel conveyance, equation \ref{eqn:cn} can be simplified as

\begin{equation}
  C_{nm} = 
  \frac{
  \kappa_n 
  }
  {
  L_{nm}
  \sqrt{| \gamma_n |}
  },
\label{eqn:cn2}
\end{equation}

Channel conveyance is calculated in several different ways depending on how a cross section is defined for a reach.  For the cross sections shown in figure \ref{fig:cxs} a single Manning's roughness coefficient is assigned for the entire section.  If a single Manning's roughness coefficient is used to define the entire section for reach $n$, then the channel conveyance is calculated according to equation \ref{eqn:conveyance}.  If a cross section does not have a constant Manning's roughness coefficient for all line segments that define the channel, as shown for example in Figure \ref{fig:cxs_rough}, then a composite conveyance is calculated by summing the individual conveyance parts for the channel as

\begin{equation}
  \kappa_n = \sum_{i=1}^{NLS} \frac{A_{n,i} \left ( \frac {A_{n,i}}{P_{n,i}}\right )^{\frac{2}{3}}}{n_{n,i}},
\end{equation}

\noindent where $NLS$ is the number of line segments used to define the channel cross section.


\subsection{External Source and Sink Flow}

\subsection{Derivation of Control-Volume Finite-Difference Equations}

A general control-volume finite-difference (CVFD) equation can be written as

\begin{equation}
  \label{eqn:cvfd}
  \sum \limits_{m \in \eta_{n}} Q_{n,m} + Q_{n,s} - Q_{STO} = 0,
  \end{equation}
  
\noindent where $\eta_{n}$ is a list of the cells connected to cell $n$, $Q_{n,m}$ is the flow rate into cell $n$ from cell $m$ ($Q_{n,m}$ is negative if flow is from cell $n$ to cell $m$) with dimensions of L$^{3}$T$^{-1}$, $Q_{n,s}$ is the flow rate of sources and sinks into cell $n$ (L$^{3}$T$^{-1}$), and $Q_{STO}$ is the change in the volume of water stored in cell $n$ (L$^{3}$T$^{-1}$).

\subsection{Channel Cross Sections}

% This figure below is for the case where roughness is the same for every line segment in the cross section 
\begin{figure}[h!tbp]
	\centering
	\includegraphics[scale=0.5]{figures/cxs.pdf}
	\caption[Schematic showing different types of channel cross sections with constant roughness.]{Schematic showing different types of cross sections with a single Manning's roughness coefficient $n$: (a) hydraulically wide cross section defined using the width input parameter; (b) hydraulically wide cross section defined using two points; (c) a rectangular cross section defined using four points, and (d) a trapezoidal cross section defined using four points.  For the hydraulically wide rectangular cross sections in (a) and (b) the model does not include any channel resistance for the vertical wetted sections corresponding to the dashed lines.  For cross sections shown in (c) and (d), if the water surface rises above the channel points, no channel resistance is included for sections above the uppermost points}
	\label{fig:cxs}
\end{figure}

% This figure below is for the case where roughness varies for each line segment in the cross section 
\begin{figure}[h!tbp]
	\centering
	\includegraphics[scale=0.5]{figures/cxs_rough.pdf}
	\caption[Schematic showing different types of channel cross sections with variable roughness.]{Schematic showing different types of cross sections with variable Manning's roughness coefficient: (a) a rectangular cross section defined using four points, and (b) a trapezoidal cross section defined using four points.  The Manning's roughness coefficient is varies by line segment.  If the water surface rises above the channel points, no channel resistance is included for sections above the uppermost points}
	\label{fig:cxs_rough}
\end{figure}

\subsection{Aquifer Interaction}

The SWF Model can be connected to an underlying Groundwater Flow (GWF) Model using a SWF-GWF Exchange.  The flow between the SWF and GWF Models is calculated based on the following expression for flow,

\begin{equation}
  Q_{n,m} = s_{n,m} A_{n,m} \acute{K}_{n,m} (h_m - h_n),
\end{equation}

\noindent where $Q_{n,m}$ is the volumetric flow rate between a surface water cell $n$ and a groundwater model cell $m$ in $L^{3}$T$^{-1}$, $s_{n,m}$ is a quadratic smoothing factor that ranges from zero when the upstream-weighted water depth is zero to a value of one when the upstream-weighted water depth is greater than or equal to a smoothening depth (presently set as $10^{-6}$), $A_{n,m}$ is the interaction area for flow ($L^2$), $\acute{K}_{n,m}$ is the leakance ($T^{-1}$) of the bed sediments calculated as the hydraulic conductivity of the bed sediments divided the thickness of the bed sediments, and $h_n$ and $h_m$ are the water surface elevation and groundwater head of the surface water cell and groundwater cell, respectively.

The interaction area for flow $A_{n,m}$ is defined differently for channel cells than for overland flow cells.  For channel cells, the interaction area is defined as

\begin{equation}
  A_{n,m} = L_{n,m} P_{up},
\end{equation}

\noindent where $L_{n,m}$ is the length ($L$) of channel $n$ overlying groundwater model cell $m$, and $P_{up}$ is the upstream-weighted wetted perimeter of the channel ($L$) calculated using either $h_n$ or $h_m$, whichever is greater.

For overland flow cells, the interaction area corresponds to the overlapping surface area between the overland flow model cell and the underlying groundwater flow model cell.  This area is provided as input by the user.  If the overland flow model grid in plan view is the same as the groundwater flow model grid, then the interaction area is simply the area of the cell; however, it is possible to use a different grid and so the interaction area must be provided by the user.

\section{Numerical Solution}

\subsection{Newton-Raphson Method}

Flow throughout the connected channel network is based on a Newton-Raphson formulation, in which the following equation is formulated and solved simultaneously for each reach \textcolor{red}{change n and m to i and j?}

\begin{equation}
\label{eqn:nr-cvfd}
\begin{split}
\left ( \sum\limits_{m \in \eta_{n}} \frac{\partial Q_{n,m}}{\partial h_n} + \frac{\partial Q_{n,s}}{\partial h_n} - \frac{\partial Q_{STO}}{\partial h_n} \right ) h^k_n + 
\sum\limits_{m \in \eta_{n}} \frac{\partial Q_{n,m}}{\partial h_m} h^k_{m} = \\
- \left ( \sum\limits_{m \in \eta_{n}} Q_{n,m} + Q_{n,s} - Q_{STO} \right ) + \\
\left ( \sum\limits_{m \in \eta_{n}} \frac{\partial Q_{n,m}}{\partial h_n} + \frac{\partial Q_{n,s}}{\partial h_n} - \frac{\partial Q_{STO}}{\partial h_n} \right ) h^{k-1}_n + \sum\limits_{m \in \eta_{n}} \frac{\partial Q_{n,m}}{\partial h_m} h^{k-1}_{m}.
\end{split}
\end{equation}


\subsection{Temporal Discretization}
Adaptive time stepping

\subsection{Sources and Sinks}

\subsection{Initial and Boundary Conditions}

A zero-depth-gradient (ZDG) boundary condition can be assigned to any reach to allow from the channel.  Flow out of the channel is calculated as

\begin{equation}
  Q = \frac{1}{n}A R^{2/3} \sqrt{S_0}
\end{equation}


\section{Integration into MODFLOW 6}

Need to update this section with perturbation approach described by Kelly
The epsilon used for derivative calculation must be small.  1.D-8 was working well until we tried the one-dimensional problem with tiny cells (200,000 cells).  With 1.D-8, a floating point exception was encountered in the dot product as part of the bicgstab calculation.  When this value was reduced to 1.D-12, the program worked as expected.

\newpage
\section{Examples}

% Ideally, each example should be written up following a similar template.  A possible template is
% background
% purpose of the test
% expectation
% problem description
% results

\subsection{Steady One-Dimensional Flow}

% background
To ensure that numerical models are working properly, results from numerical simulations are commonly compared with results from an analytical solution.  To demonstrate and test the SWF Model, an analytical solution was developed for simple one-dimensional steady flow.  An analytical solution can be obtained for the following mathematical governing equation

\begin{equation}
  \frac{\partial h}{\partial t} = \frac{\partial}{\partial x} 
  \left ( \frac{h^{5/3}}{n \left | \frac{\partial h}{\partial x} \right |^{1/2}} 
  \frac{\partial h}{\partial x} \right ) = 0 .
  \label{eqn:gov_1d}
\end{equation}

\noindent Equation~\ref{eqn:gov_1d} describes flow for a hydraulically wide rectangular channel, in which the wetted perimeter is equal to the channel width, and the bottom is flat with an elevation of zero.  Under these conditions, the water surface elevation $h$ is equal to the water depth $d$.  For flow between two locations, $x_0$ and $x_1$, with prescribed stages of $h_0$ and $h_1$, respectively, the analytical solution for $h$ as a function of $x$ is

\begin{equation}
  h = \left [ \left (1 - \rho \right ) h^{\frac{13}{3}}_{0} + \rho h^{\frac{13}{3}}_{1} \right ]^{\frac{3}{13}} ,
  \label{eqn:asoln_1d}
\end{equation}

\noindent where

\begin{equation}
  \rho \equiv \frac{x - x_0}{x_1 - x_0} .
  \label{eqn:rho_defined_x}
\end{equation}

\noindent Equation~\ref{eqn:asoln_1d} can be solved easily to produce steady channel stage profiles for comparison with model results.

Under these conditions the flow rate $Q$ can be calculated as

\begin{equation}
  Q = \frac{1}{n} \left ( 
    \frac{3}{13}
    \frac{h^{\frac{13}{3}}_{0} - h^{\frac{13}{3}}_{1}}{x_1 - x_0}
  \right )^{1/2}.
  \label{eqn:q_calc}
\end{equation}


% purpose and expectation
In this example, a simple one-dimensional numerical model is used to simulate flow and the water surface profile between two prescribed stage boundaries.  Results from the numerical model should be in agreement with the analytical solution.  Some minor differences are expected due to upstream weighting and discretization errors in the numerical model.  These differences between the numerical model and the analytical solution should decrease with increases in model grid resolution.

%problem description
One dimensional steady flow is simulated using two different approaches.  The DISV1D discretization is used for the first approach to represent flow in a channel.  The DIS2D discretization, with a single row, is used to represent flow in overland conditions.  In this example, a model domain extending 110 km in the x direction is divided into 501 cells.  Prescribed stage conditions are assigned to the first and last cells.  A stage value of 10.0 m is assigned to the first cell and a value of 1.0 m is assigned to the last cell.  Steady conditions are solved using a stage tolerance of $10^{-8}$ m.

%results
The comparison between the analytical solution (equation~\ref{eqn:asoln_1d}) and the channel and overland flow models are shown in figure~\ref{fig:oned-results}.  The upper plot shows the stage profile (for every tenth point) and the close agreeement between the analytical and numerical solutions.  Although there is good agreement between the numerical models and the analytical solution, the lower plot shows that maximum errors between the models and the analytical solution are about 0.4 m.  This error in the numerical models is due to spatial discretization.  An increase in spatial resolution results in a better match with the analytical solution.  Using equation~\ref{eqn:q_calc} the flow under these conditions is 0.212805 $m^3/s$.  The simulated flow for the channel and overland models is in error by about 0.3 percent. Simulated flow for the channel and overland models 0.213465 and 0.213428 $m^3/s$, respectively.  

\begin{figure}[h!tbp]
	\centering
	\includegraphics[scale=1.0]{figures/oned.png}
	\caption[Results for one-dimensional flow.]{Results for one-dimensional flow.  Two different Surface Water Flow (SWF) Models are shown.  The channel model represents flow using the one-dimensional DISV1D discretization.  The overland model represents flow using the two-dimensional DIS2D discretization with a single row.}
	\label{fig:oned-results}
\end{figure}

\newpage
\subsection{Steady Radial Flow}

% background
To further demonstrate and test the SWF Model, an analytical solution was also developed for steady one-dimensional radial flow.  Following the previous example, if the bed is level at elevation $z = 0$ and the depth is equal to the stage ($d = h$), then the governing equation for one-dimensional radial surface water flow is

\begin{equation}
  \frac{\partial h}{\partial t} = \frac{1}{r} \frac{\partial}{\partial r} 
  \left (r \frac{h^{5/3}}{n \left | \frac{\partial h}{\partial r} \right |^{1/2}} 
  \frac{\partial h}{\partial r} \right ) = 0 .
\end{equation}

\noindent For flow between two prescribed stage boundaries, with stages of $h_0$ and $h_1$ at radial distances of $r_0$ and $r_1$, respectively, the analytical solution for $h$ as a function of radial distance $r$ is

\begin{equation}
  h = \left [ \left (1 - \rho \right ) h^{\frac{13}{3}}_{0} + \rho h^{\frac{13}{3}}_{1} \right ]^{\frac{3}{13}} ,
  \label{eqn:radial-flow-analytical}
\end{equation}

\noindent where

\begin{equation}
  \rho \equiv \frac{\frac{1}{r_{0}} - \frac{1}{r}}{\frac{1}{r_{0}} - \frac{1}{r_{1}}} .
  \label{eqn:rho_defined}
\end{equation}

% purpose and expectation
In this example, two different model grids are used to simulate steady overland flow and the water surface profile between two prescribed stage boundaries.  Results from the numerical models should be in agreement with the analytical solution.  Some minor differences are expected due to upstream weighting and discretization errors in the numerical model.

%problem description
A structured two-dimensional model grid is represented using the DIS2D discretization package (figure~\ref{fig:oned-structured-grid}).  The DIS2D grid consistes of 151 rows and 151 columns.  Constant heads are assigned to the group of nine model cells in the middle of the domain and around the perimeter of the domain.  The constant head values are calculated using the analytical solution for radial flow (equation \ref{eqn:radial-flow-analytical}).

\begin{figure}[h!tbp]
	\centering
	\includegraphics[scale=0.70]{figures/oned-structured-grid.pdf}
	\caption[Structured model grid used to simulate radial flow.]{Structured model grid used to simulate radial flow.  Grid consists of 151 rows and 151 columns.  The model domain spans 162,000 meters in both the x and y directions.  Constant head cells are shown in blue.  Head values for the constant head cells are calculated using the radial flow analytical solution (equation \ref{eqn:radial-flow-analytical}).}
	\label{fig:oned-structured-grid}
\end{figure}

A voronoi model grid, represented using the DISV2D discretization package, is also used to simulate radial overland flow (figure~\ref{fig:oned-voronoi-grid}).  The voronoi model grid was constructed using an inner radius of 5000 m and an outer radius of 80,000 m.  Cells within the gridded domain are no larger than $(5000 m)^2$ in area.  Constant heads are assigned to the ring of innermost cells and outmost cells.  Head values for the constant head cells are calculated using the radial flow analytical solution (equation \ref{eqn:radial-flow-analytical}).

\begin{figure}[h!tbp]
	\centering
	\includegraphics[scale=0.70]{figures/oned-voronoi-grid.pdf}
	\caption[Voronoi unstructured model grid used to simulate radial flow.]{Voronoi unstructured model grid used to simulate radial flow.  Grid consists of 2392 model cells and 6169 vertices.  The radius of the inner circle is 5000 meters, and the radius of the outer circle is 80,000 meters.  Constant head cells, shown in blue, are assigned to the innermost cells and the outermost cells.  Head values for the constant head cells are calculated using the radial flow analytical solution.}
	\label{fig:oned-voronoi-grid}
\end{figure}

%results
The comparison between the analytical solution (equation~\ref{eqn:radial-flow-analytical}) and the simulation with the structured model grid is shown in figure~\ref{fig:oned-structured-results}.  The comparison between the analytical solution (equation~\ref{eqn:radial-flow-analytical}) and the simulation with the voronoi model grid is shown in figure~\ref{fig:oned-voronoi-results}.  In both cases, the simulation results are in good agreement with the analytical solution.

\begin{figure}[h!tbp]
	\centering
	\includegraphics[scale=0.9]{figures/oned-structured-results.pdf}
	\caption[Simulation results from the structured grid model.]{Simulation results from the structured grid model.  Black dotted contours are for the numerical model and the red solid contours are for the analytical solution.  The contour interval is 0.2 meters.}
	\label{fig:oned-structured-results}
\end{figure}

\begin{figure}[h!tbp]
	\centering
	\includegraphics[scale=0.9]{figures/oned-voronoi-results.pdf}
	\caption[Simulation results from the voronoi grid model.]{Simulation results from the voronoi grid model.  Black dotted contours are for the numerical model and the red solid contours are for the analytical solution.  The contour interval is 1.0 meters.}
	\label{fig:oned-voronoi-results}
\end{figure}

\newpage
\subsection{Axisymmetric Overland Flow}

The axisymmetric flopy problem developed by~\cite{lal2001}, and used by~\cite{hughes2015} to test the SWR Process for MODFLOW-2005, is used to test the SWF Model for MODFLOW 6.  The problem is based on the transient decay of a circular mound of surface water.

\begin{figure}[h!tbp]
	\centering
	\includegraphics[scale=0.70]{figures/axi-results.png}
	\caption[Simulation results for the axisymmetric overland flow problem.]{Simulation results for the axisymmetric overland flow problem.}
	\label{fig:axi-results}
\end{figure}

\newpage
\subsection{Tilted V-Catchment}

The MODFLOW 6 SWF Model was used to simulate the tilted V-catchment problem described by \cite{digiammarco1996}.  This tilted V-catchment problem has been used by~\cite{VanderKwaak1999},~\cite{panday2004} and by~\cite{hughes2015}, for example, to test numerical models of two-dimensional overland in response to a rainfall event.

The configuration of the tilted V-catchment is shown in Figure~\ref{fig:vcatch-surface}.  The left and right sides of the domain consist of sloping flat planes.  The flat planes are tilted to the south and toward the middle of the domain, forming a V-shaped catchment.  A slope of 0.05 is used for the x direction, and a slope of 0.02 is used for the y direction.  Located between the two sloping planes is a 20-m wide stream channel that slopes from north to south.  Two different Mannings roughness coefficients are used.  A value of 0.015 is assigned to the sloping planes and a value of 0.15 is assigned to the stream channel.  These values are not realistic, in that roughness coefficients are generally lower within a stream channel compared to sloping surfaces, but they were selected as a difficult test for simulating overland flow.  A zero-depth gradient boundary condition assigned as the southernmost part of the stream provides the sole outlet for the domain.

\begin{figure}[h!tbp]
	\centering
	\includegraphics[scale=0.75]{figures/vcatch-surface.png}
	\caption[Configuration of the tilted v-catchment example.]{Configuration of the tilted v-catchment example.  Contours represent land surface.  A slope of 0.05 is used for the x direction, and a slope of 0.02 is used for the y direction.  A 20-m wide stream channel, shown in gray, slopes from north to south.}
	\label{fig:vcatch-surface}
\end{figure}

The domain is discretized into a regular model grid consisting of 50 rows and 81 columns.  Each cell is square with a dimension of 20 m.  Rainfall is uniformly applied to the domain at a rate of $3 \times 10^{-6}$ for a period of 90 minutes.  A second 90-minute period with no rainfall is also represented to simulate drainage of the domain in response to the rainfall event.  Following the settings used by~\cite{panday2004} adaptive time stepping is used with an inital time step of 5 s and a maximum time step of 100 s.  Time steps increase or decrease by a factor of two in response to solver behavior.

Results from the MODFLOW 6 SWF Model simulation are compared in Figure~\ref{fig:vcatch} with the simulations results of~\cite{digiammarco1996},~\cite{VanderKwaak1999},~\cite{panday2004} and~\cite{hughes2015}.  In general, the model results from MODFLOW 6 compare favorably with the results from these other models; however, rates of simulated outflow from the MODFLOW 6 simulation rise and fall more quickly than for the other models.

\begin{figure}
	\centering
	\includegraphics[scale=0.75]{figures/vcatch-results.png}
	\caption[Simulated discharge for the tilted v-catchment example.]{Simulated discharge for the tilted v-catchment example.  Dashed blue line separates the first 90-minute period with rainfall from the second 90-minute period with no rainfall.}
	\label{fig:vcatch}
\end{figure}

\subsection{Tilted V-Catchment with Aquifer Exchange}

\cite{panday2004} modified the tilted v-catchment problem to include an underlying aquifer.  Flow in their underlying aquifer was represented using Richards equation in order to represent unsaturated flow.  Given that MODFLOW 6 does not have a Richards formulation, this problem was represented using a single aquifer layer with the unconfined flow equation.

\subsection{Modified Pinder-Sauer}
Pinder-Sauer problem is a channel flowing over an aquifer.  A sinusoidal inflow at the top of the channel propogates through the domain.  Lal (xxx) came up with an analytical solution that is compared to fluctuations in the simulated flow in the channel as well as fluctuations in simulated stage in the channel.

\bibliographystyle{plain}
\bibliography{swfref} 

\end{document}